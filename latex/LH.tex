\documentclass[10pt]{article} 
\usepackage[utf8]{inputenc}

% 10pt permet de déterminer la taille de la fonte
\usepackage[utf8]{inputenc}

\usepackage{amssymb,amsmath,amsthm,amsfonts, bbm} 
% il s'agit de différents packages utilisés pour écrire des mathématiques en Latex

\usepackage{geometry}
\geometry{hmargin=2.3cm,vmargin=2.5cm}
% ce package permet de spécifier la taille des marges (h: horizontales; v: verticales)
\usepackage[svgnames]{xcolor}
\usepackage{array}
\usepackage{graphicx,graphics}
\usepackage{caption}
\usepackage{subcaption}
\usepackage{stmaryrd}
%\usepackage{minted}
\usepackage{dsfont}
\usepackage{titling}
\usepackage{pdfpages}
\usepackage{amsmath}
\usepackage{amsfonts}
\usepackage{epsfig}
\usepackage{listings}

\usepackage[section]{placeins}
% packages utilisés pour faire des tableaux, graphiques et mise en page

\usepackage{url}
\lstset{language = Python,
inputencoding=utf8,
    extendedchars=true,
    literate={á}{{\'a}}1 {ã}{{\~a}}1 {é}{{\'e}}1 {è}{{\`e}}1 {à}{{\`a}}1 {ô}{{\^o}}1 {ù}{{\`u}}1,
    backgroundcolor=\color{WhiteSmoke},   
    commentstyle=\color{DarkOrange},
    keywordstyle=\color{NavyBlue},
    numberstyle=\tiny\color{LightSlateGrey},
    stringstyle=\color{MediumVioletRed},
    basicstyle=\footnotesize\color{DarkSlateGrey},
    breakatwhitespace=false,         
    breaklines=true,                 
    captionpos=b,                    
    keepspaces=true,                 
    numbers=left,                    
    numbersep=5pt,                  
    showspaces=false,                
    showstringspaces=false,
    showtabs=false,                  
    tabsize=2,}



\title{Compte-rendu Projet Python: \\ Création d'une intelligence artificielle pour le Tic Tac Toe $9\times9$ }
\author{Pierre Eustache - Laure-Hélène Genuyt }
\date{21 mai 2018}

\begin{document}

\maketitle

%\section{Objectifs du rapport}
 %Le rapport doit contenir les éléments suivants :

%\begin{itemize}
 %   \item instructions pour exécuter le programme

%\item description d’un point de design de code
%\item
%description d’un algorithme du programme et son coût
%\item
%description d’une difficulté qui vous a fait perdre beaucoup de temps (hors celles liées à votre apprentissage de la programmation)
%\item
%description d’une contribution originale
%\item
%description des résultats obtenus
%\item
%prolongements et applications envisageables

%\end{itemize}
\section{Introduction}
L'objectif de ce projet a été de créer une intelligence artificielle pour le jeu du Tic Tac Toe $9\times9$ (aussi appelé morpionception, ou Ultimate Tic Tac Toe). 
\paragraph{}


    \begin{enumerate}
    
\item \textbf{Règles du jeu}\\

Il s'agit d'un jeu de plateau, composé d'une grille de 3 cases par 3, dont chacune des cases contient elle-même une grille de 3 par 3. Dans la suite, on se référera à la "grande grille" et aux "petites grilles" pour distinguer ces deux niveaux de jeu.\\
L'objectif du jeu est de remporter 3 cases de la grande grille, en y apposant son symbole (une croix ou un cercle). Pour cela, les joueurs jouent au niveau des petites grilles, et cherchent à les remporter en alignant 3 de leurs symboles. Une fois la petite grille gagnée, on ne peut plus y jouer, et elle devient elle-même une case "croix" ou "cercle". \\
Le jeu présente cependant une subtilité qui le rend intéressant: chaque joueur doit jouer dans la case de la grande grille correspondant à la case de la petite grille où a joué l'autre joueur au tour précédent.


%\paragraph{}
%\includegraphics[width=0.25\textwidth]{tour1.png}
%\caption{Joueur A a joué dans la case en haut à gauche d'une petite grille} \\
%\includegraphics[width=0.25\textwidth]{tour2.png}
%\caption{Joueuer B doit donc jouer dans la case en haut à gauche de la grande grille} \\
%\includegraphics[width=0.25\textwidth]{tour3.png}
%\caption{Comme B a joué dans la case en haut à droite de sa petite grille, A doit jouer dans la case correspondante de la grande grille} \\
%\end{figure}


\begin{figure}
 \begin{minipage}[t]{0.25\linewidth}
  \centering\epsfig{figure = tour1.png,width=\linewidth}
  \caption{Dans cet exemple, A a joué dans la case en haut à gauche de sa petite grille.}
 \end{minipage} \hfill
 \begin{minipage}[t]{0.25\linewidth}
  \centering\epsfig{figure = tour2.png,width=\linewidth}
  \caption{Au tour suivant, B doit donc jouer dans la case en haut à gauche de la grande grille}
  \end{minipage} \hfill
 \begin{minipage}[t]{0.25\linewidth}
  \centering\epsfig{figure = tour3.png,width=\linewidth}
  \caption{Comme B a joué dans la case en haut à droite de sa petite grille, A doit ensuite jouer dans la case correspondantes de la grande grille (etc.)}
 \end{minipage}
\end{figure}


\paragraph{}
Lorsqu'un joueur renvoie l'autre dans une case de la grande grille déjà gagnée, celui-ci peut jouer où il le souhaite. \\

Le jeu se termine dès qu'un joueur a réussi à aligner 3 croix ou 3 cercles dans des cases de la grande grille. 


\paragraph{}
\begin{figure}[h!]
\centering
\includegraphics[width=0.5\textwidth]{exemple.png}
\caption{Exemple d'une partie de Tic Tac Toe $9\times9$}
\end{figure}

\item \textbf{Difficultés du jeu} \\

La création d'une intelligence artificielle pour ce jeu, en apparence relativement simple, pose quelques problèmes. 
En effet, contrairement au jeu du Tic Tac Toe classique, qui se termine par un match nul à moins d'une erreur d'inattention de la part d'un des joueurs, il est difficile de trouver une stratégie gagnante à ce jeu, et de bien en saisir tous les mécanismes.
\paragraph{}
De plus, la création d'une intelligence artificielle suppose de pouvoir évaluer chaque coup et de déterminer la valeur d'une position. Or, cela ne va pas de soi dans ce jeu, qui se joue à plusieurs niveaux: un coup doit-il être évalué à l'aune de ce qu'il apporte au joueur dans la petite grille où il est joué? \\
Ou faut-il davantage tenir compte de la case de la grande grille où il renvoie l'adversaire? 

\paragraph{}
\item \textbf{Méthode suivie}

\begin{itemize}
    \item Nous avons tout d'abord réfléchi à une manière optimale de coder une partie, et plus particulièrement l'état de la grille à un instant donné. Nous avons décidé de répresenter  la grande grille comme un tableau $3\times3$ de tableaux $3\times3$ (les petites grilles). Chacune des 81 cases est donc référencée grâce à un quadruple indice. \\
    \underline{Ex}: la position (1,1,3,3) fait référence à la petite case en bas à droite de la grande case en haut à gauche. \\ 
    Notre tableau est composé de 0, de 1 et de 2, où 0 représente une case vide, 1 une case contenant une croix, et 2 une case contenant un cercle. \\
    Nous avons ensuite créé un algorithme permettant de coder ce tableau en binaire, afin d'optimiser l'espace pris par une grille, mais cela ne nous as finalement pas servi par la suite.
   \paragraph{}
    \item Nous nous sommes ensuite servis du module \textit{tkinter} pour créer une interface graphique affichant l'état de la grille à un instant donné, à partir du tableau décrit précédemment. 
    \paragraph{}
    \item Afin de créer notre intelligence artificielle, nous avons essayé d'évaluer la valeur d'un coup grâce à une analyse heuristique. 
    En effet, contrairement à un jeu d'échecs, où la valeur d'un coup peut être assimilé à la valeur du pion qu'il permet de prendre, il est difficile de trouver une valeur objective d'un coup au Tic Tac Toe $9\times9$. Nous avons donc crée une fonction prenant en compte un certain nombre d'indicateurs qui nous semblaient entrer dans l'évaluation d'un coup, et attribuant à chacun une pondération - déterminée d'après notre avis subjectif et notre expérience du jeu dans un premier temps. \\
    Ces indicateurs sont: 
    \begin{itemize}
        \item la case que l'on compte jouer se trouve-t-elle dans un coin? 
        \item se trouve-t-elle au centre d'une case? 
        \item se trouve-t-elle au bord d'une case? 
        \item combien de symboles a-t-on dans la grille où on renvoie l'adversaire?
        \item combien l'adversaire possède-t-il de symboles dans la grille où on le renvoie?  
        \item la case où on renvoie l'adversaire est-elle déjà gagnée?  
        \item cette case est-elle vide? 
        \item ce coup renvoie-t-il l'adversaire dans une case qu'il peut gagner? 
        \item ce coup permet-il de remporter la case? 
        \item ce coup permet-il de gagner le jeu? 
    \end{itemize}
\paragraph{}
Ces indicateurs ont une pondération positive si on juge qu'ils sont en faveur du joueur, négative si on juge le contraire. \\
On part par exemple du principe qu'il est plus avantageux de jouer au centre que dans un coin, et plus avantageux de jouer dans un coin que de jouer au bord. \\
On considère également qu'il vaut renvoyer l'adversaire dans une case où il possède peu de symboles, et qu'il faut surtout éviter de l'envoyer dans une case déjà gagnée (ce qui lui permettrait de jouer n'importe où par la suite). 
\paragraph{}
\item Une fois cette heuristique créée, nous avons codé les règles du jeu (définition des coups possibles, enchaînement des actions, etc.), avant de créer une intelligence artificielle jouant aléatoirement selon ces règles (IA de type \textit{dummy} dans le code). 
\paragraph{}
\item Nous avons ensuite implementé un algorithme génétique dans le but d'améliorer nos pondérations, et de créer une intelligence artificielle jouant selon ces pondérations. \\
La performance de ces intelligences artifcielles (de type \textit{eurist} dans le code) a été evaluée en fonction de leur taux de victoire et de défaite contre l'intelligence artificielle jouant aléatoirement. 
L'algorithme génétique se décompose en plusieurs étapes: 
\begin{itemize}
    \item créer un lot de joueurs, jouant selon des pondérations déterminées de façon subjective dans un premier temps. 
    \item faire jouer ces joueurs contre notre intelligence artificielle aléatoire, en enregistrant leurs scores. 
    \item une fois le tournoi terminé, trier les joueurs selon leur score, et en garder la meilleure moitié. 
    \item muter le reste des joueurs, en modifiant de manière aléatoire leurs pondérations. 
    \item avec ce nouveau lot de joueurs (meilleure moitié et joueurs mutés), recommencer à partir de la deuxième étape.   
\end{itemize}

\item Au bout de ... générations, nous avons obtenu les résultats suivant: 
%insérer résultats: je sais pas si tu les as gardés où s'il faut refaire tourner le programme
\\ 
On voit donc que notre intelligence artificielle générée par cet algorithme génétique présente une très nette amélioration par rapport à notre intelligence artificielle aléatoire. \\
\paragraph{}
\item Cette intelligence artificielle heuristique présente cependant une faiblesse majeure, qui est qu'elle n'a aucune profondeur dans son évaluation des coups: elle joue simplement selon des pondérations permettant d'évaluer la valeur du coup qu'elle va jouer \\
Nous avons donc implémenté un algorithme minimax, permettant à l'IA de voir à une profondeur de 3 coups. \\
Cet algorithme, que l'on peut représenter sous forme d'un arbre, examine toutes les évolutions possibles du jeu à partir de la configuration actuelle,et attribue à chacune de ces feuilles une valeur (selon les pondérations de la meilleure intelligence artificielle générée par l'algorithme génétique). \\
On considère que l'adversaire (qui joue au coup d'après), va choisir le coup qui l'amène à la situation la plus bénéfique pour lui, et donc la plus néfaste pour nous. \\
La valeur d'un noeud de l'adversaire correspond donc au minimum des feuilles évaluées précédemment. La valeur de chacun de nos coups possibles va donc être le maximum des valeurs des noeuds de l'adversaire, et on va jouer le coup possédant la plus haute valeur (il s'agit donc du coup qui nous permettra d'atteindre la situation de jeu la plus optimale pour nous dans 3 coups, en sachant que l'adversaire choisira toujours le coup qui lui est le plus avantageux).

\paragraph{}
\begin{figure}[h!]
\centering
\includegraphics[width=0.7\textwidth]{minimax.png}
\caption{ }{Exemple d'arbre minimax (les carrés représentent ici les noeuds joueurs et les ronds les noeuds opposants)}
\end{figure}


Cet algorithme permet à l'intelligence artificielle (définie dans le code comme étant de type \textit{mini}) de gagner en performance. \\
Contre l'intelligence artificielle aléatoire, on a en effet observé, sur 200 matchs: 
\begin{itemize}
    \item 183 victoires
    \item 14 matchs nuls
    \item 3 défaites 
    % il me semble que c'était ça...
\end{itemize}

\paragraph{}
\item Une fois ces étapes réalisées, nous avons complété les classes \textit{Player} et\textit{Game}, permettant chacune de définir les fonctions et les propriétés d'un joueur et d'une partie. 

\item In fine, notre programme permet donc: 
\begin{itemize}
    \item faire jouer deux intelligences artificielles (de type \textit{dummy}, \textit{eurist} ou \textit{mini}) l'une contre l'autre
    \item de jouer soi-même contre une intelligence artificielle de l'un de ces 3 types, sur une interface graphique adaptée.

\end{itemize}
\paragraph{}
$\Longrightarrow$ On a donc bien réussi à créer une intelligence artificielle pour le jeu du Tic Tac Toe $9\times9$, contre laquelle on peut jouer dans une interface graphique satisfaisante. \\
Si notre intelligence artificielle la plus satisfaisante (celle jouant selon l'algorithme minimax) est extrêmement performante contre une intelligence artificielle aléatoire, et constitue donc un réel succès sur ce point, elle s'avère cependant un peu faible face à un humain. \\
Cela est probablement dû à la difficulté d'évaluer simplement une situation de la grille à un moment donné, ainsi qu'à la difficulté qu'il y a à trouver une stratégie gagnante à ce jeu somme toute beaucoup moins simple qu'il n'y paraît. 


\end{itemize}

%\item \textbf{Instructions pour utiliser le programme}
%Une fois le programme exécuté, pour lancer une partie contre une intelligence artificielle, il suffit de:
%\begin{itemize}
%    \item 
%\end{itemize}

\paragraph{}
\item \textbf{Description plus détaillée de l'algorithme génétique}

\begin{itemize}
    \item L'algorithme génétique consiste à répéter plusieurs fois les opérations suivantes: 
    \begin{itemize}
        \item créer une génération de joueurs
        \item les faire jouer plusieurs fois contre l'intelligence artificielle aléatoire  
        \item garder la meilleure moitié
        \item "muter" l'autre moitié, en modifiant aléatoirement les pondérations auxquelles ces joueurs obéissent 
        \item recommencer avec cette nouvelle génération
    \end{itemize}
    Une étape de cet algorithme (soit une \textit{génération}, correspond à la fonction suivante: 
    \lstinputlisting[language = Python, firstline = 41, lastline = 47]{genetique.py}
Cette fonction \textit{génération} fait elle-même appel à plusieurs autres fonctions, que nous allons détailler maintenant. 

\item La fonction \textit{génération} prend plusieurs arguments: 
\begin{itemize}
    \item n, qui correspond à la taille de l'échantillon de joueurs que nous allons considérer, et qui est ici fixé à 30
    \item nb, qui correspond au nombre de matchs que chacun de ces joueurs va faire contre l'intelligence artificielle aléatoire, et qui est ici fixé à 100
    \item N, qui correspond à la liste des joueurs 
\end{itemize}

\item À la première génération, N est encore vide. On crée donc un lot de joueurs, en faisant appel à la fonction \textit{create\_batch}: 
\lstinputlisting[language = Python, firstline = 10, lastline = 14]{genetique.py}
Cette fonction prend en argument n, la taille du lot de joueurs, et renvoie une liste contenant n éléments de la classe \textit{Player}, la classe qui définit les propriétés et les fonctions d'un joueur. 

 Une fois ce lot créé, on fait appel à la fonction \textit{randomize} pour générer des pondérations aléatoires: 
\lstinputlisting[language = Python, firstline = 16, lastline = 20]{genetique.py}
Cette fonction prend en argument une liste de joueurs et, pour chacun des joueurs de cette liste, leur attribue une tableau de pondérations générées aléatoirement. 

\item L'étape précédente n'est réalisée qu'à la première génération, si N est vide. Sinon, on passe directement à l'étape suivante: le tournoi entre chacun des joueurs de N et l'intelligence artificielle aléatoire. 
Pour cela, on fait appel à la fonction \textit{dummy\_tournament}: 
\lstinputlisting[language = Python, firstline = 49, lastline = 67 20]{genetique.py}

Cette fonction prend deux arguments: 
\begin{itemize}
    \item players, une liste de joueurs
    \item n, le nombre de fois que chaque joueur va jouer contre l'intelligence artificielle, et qui est ici fixé à 60. 
\end{itemize}
On crée tout d'abord un joueur b, de type \textit{dummy}, et qui correspond à l'intelligence artificielle aléatoire contre laquelle nos joueurs vont jouer.\\
On crée également j, un compteur du nombre de joueurs qui ont déjà effectué leurs matchs, afin de pouvoir suivre l'avancement du programme. \\
Ensuite, pour chaque joueur: 
\begin{itemize}
    \item On augmente j de 1 et on l'affiche, pour suivre l'avancée du programme
    \item On réalise les opérations suivantes n fois: 
    \begin{itemize}
        %\item je ne comprend pas bien ce que tu as fais avec l... 
        \item si n est impair, on décide que l'intelligence artificielle aléatoir jouera en deuxième
        \item si n est pair, elle commencera à jouer. 
        \item on joue ensuite la partie
        \item si le vainqueur est notre joueur étudié, on augmente son score de 1
    \end{itemize}
La fonction renvoie enfin le score de chaque joueur. 
\end{itemize}


\item Une fois ce tournoi effectué et les scores des joueurs conservés en mémoire, on fait appel à la fonction \textit{mutate} pour créer une nouvelle génération de joueurs. 
\lstinputlisting[language = Python, firstline = 22, lastline = 32]{genetique.py}
Cette fonction prend en argument une liste de joueurs, et la "mute" pour créer une nouvelle génération de joueurs. 
\begin{itemize}
    \item On commence par couper la liste de joueurs en deux, en les triant selon leurs scores. Une fois ce tri effectué, on remet les scores à zéro. \item Ensuite, pour chaque joueur de la première moitié de la liste (donc les joueurs ayant obtenu les moins bons scores), on effectue les opérations suivantes: 
    \begin{itemize}
        \item on attribue à p, au départ un tableau vide, le tableau des pondérations de ce joueur
        \item on applique la fonction \textit{mutation} à ce tableau de pondérations: 
        \lstinputlisting[language = Python, firstline = 36, lastline = 39]{genetique.py}
        Cette fonction prend en argument un tableau de pondérations, le remplace par un tableau de même taille généré aléatoirement, et renvoie ce nouveau tableau. 
        \item dans la liste des joueurs, on met à la place du joueur considéré un joueur dont les pondérations sont égales à la somme de ses pondérations de départ et des pondérations mutées.  
    \end{itemize}
On obtient alors une nouvelle liste de joueurs, dont une moitié à vu ses pondérations modifiées, et on peut appliquer à nouveau notre fonction \textit{generation} à cette liste. 
\end{itemize}

Lorsqu'on observe les scores des joueurs de chaque génération, on s'aperçoit que chaque génération est plus efficace (i.e. obtient de meilleures scores face à l'intelligence artificielle aléatoire). 

\item Coût de l'algorithme génétique
\par Bien qu'il est difficile d'estimer le coût d'une partie en elle-même, on peut calculer le coût d'une génération entière.
On note dans la suite : 
\begin{itemize}
\item $N$ le nombre d'individus
\item $m$ le nombre de matchs
\item $f$ le nombre de facteurs pris en compte dans l'heuristique
\item $d$ la profondeur de calcul du minimax
\end{itemize}
Le calcul demeure assez simple. On suppose dans la suite que l'accès aux éléments d'une liste en python est linéaire de sa taille, et qu'une partie est jouée en temps constant. Calculons les participations individuelles de chaque élément.
\begin{itemize}
\item La création est $\mathcal{O}(N)$.
\item Le calcul sur une couche de l'heuristique est $\mathcal{O}(f)$ car la grille est de taille constante.
\item Le tri de la liste se fait en $\mathcal{O}(N.ln(N))$
\item Parcourir la liste en faisant jouer chaque joueur est $\mathcal{O}(N.m.P)$ avec $P$ le temps d'une partie.
\item La dépendance selon $d$ est exponentielle, et dépend du nombre de facteurs à vérifier sur chaque niveau $\mathcal{O}(f^d)$
\item Enfin la mutation se fait en temps linéaire selon $f$ et $N$
\end{itemize}
\par On en déduit un total de $$\boxed{\mathcal{O}(N.m.f^d+N.ln(N))}$$ 
\end{itemize}







\end{enumerate}
\end{document}
